\chapter{Conclusions and Future Work}
\label{discussion}
\index{Conclusions and Future Work%
	@\emph{Conclusions and Future Work}}%


While our results do not show perfect location tracking of a user, we believe that they do make it clear that Honeycomb is a viable product for indoor location estimation, provided an adequate number of access points and fingerprint polling time. However, in performing our tests we discovered that fingerprinting itself can be a painful process, particularly with longer poll times. One way to alleviate this pain is to decrease fingerprint density in such a way that supports only the minimum viable precision necessary at a given location. Thus, fingerprint density becomes a knob that one can turn to fine tune an individual deployment of Honeycomb, and must be decided on a case by case basis. 

While this paper has proven Honeycomb's effectiveness as a product, there are still many things that must be done in order to make Honeycomb market ready. Most importantly, the user interfaces need significant work, particularly the web interface for viewing a user track, which is currently a table of timestamps and fingerprints. The result maps in this paper were created manually, but a programmatic rendering of these maps in the web interface would be preferable. 

Additionally, the web interface currently only has the ability to view each user track individually. In order to make this information useful, an aggregating a reporting function is needed which can provide information about trends in user tracks rather than data about any specific track. With this information, a site administrator could more effectively manage their space in order to maximize visibility and flow. 

However, despite shortcoming in user interfaces, we believe that Honeycomb's current incarnation represents the backbone of a viable location estimation product, and that with only relatively minor adjustments could potentially be employed in real world scenarios. 